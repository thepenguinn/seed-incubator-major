\documentclass[../weekly.tex]{subfiles}

\begin{document}

\letDaniel

\subsection{Monday: Dec 01 2025}

Today, started designing implementable version of core block.

\subsubsection{Core Block}

See figures \ref{fig:cbRoughSketches} and \ref{fig:cbStructCAD} for the
current progress.

\begin{center}
    \begin{tabularx} {\textwidth} {
            >{\centering \arraybackslash}m{0.49\textwidth}
            >{\centering \arraybackslash}m{0.49\textwidth}
        }

        \includegraphics [
            width = 0.48\textwidth,
            % angle = -90,
        ] {pics/rough_cb.jpg}
        &
        \includegraphics [
            width = 0.48\textwidth,
            % angle = -90,
        ] {pics/structural_cb_cad.png}

        \\
        &
        \\

        \captionof{figure} {Rough sketches of core block.}
        \label{fig:cbRoughSketches}
        &
        \captionof{figure} {Structural beams for the core block implemented in CAD.}
        \label{fig:cbStructCAD}

        \\

    \end{tabularx}
\end{center}

\subsubsection{Air Moisture Cycler and Exhaust Branches Interface with Mid Layer}

The incubator area should be isolated from the external world. AMC's interface with the
mid layer is one of the places where there need to be made cutouts in the mid layer.
The mid layer also should have some kind of attaching for the exhaust branches. See
figure \ref{fig:amcEbRough} for a rough sketch.

\begin{figure}
    \centering
    \includegraphics [
        width = 0.5\textwidth,
    ] {pics/rough_amc_eb.jpg}
    \captionof{figure} {Rough sketches showing the placement of air moisture cycler and exhaust branches.}
    \label{fig:amcEbRough}
\end{figure}

\subsubsection{Attachment of Mid Layer to Plywood Frame}

\paragraph{PVC Foam} is needed to be attached via nails, since glue won't be enough to secure it
in place.

\paragraph{Polycarbonate} can be screwed down to the frame. Then they can be waterproofed using
silicone adhesives.

\subsection{Tuesday: Dec 02 2025}

\subsubsection{Core Block Walls}

CAD design of the core block walls have completed. See figure \ref{fig:coreBlockWalls} for the
CAD design.

\begin{figure}
    \centering
    \includegraphics [
        width = 0.5\textwidth,
    ] {pics/core_block_walls.png}
    \captionof{figure} {CAD design of core block walls.}
    \label{fig:coreBlockWalls}
\end{figure}

\subsection{Wednesday: Dec 03 2025}

\subsubsection{Frame and Panel Design}

There will be two parts for the frame, one that is made up of PVC foam, and the other that
is 3D printed. The PVC foam part will give the necessary thermal insulation. And the printed
part will facilitate the mount for motor and limit switches. In addition it will give the
structural rigidity to the entire frame. See figures \ref{fig:cbFrameRoughOne} and \ref{fig:cbFrameRoughTwo}
for some rough sketches of the frame.

\begin{center}
    \begin{tabularx} {\textwidth} {
            >{\centering \arraybackslash}m{0.49\textwidth}
            >{\centering \arraybackslash}m{0.49\textwidth}
        }

        \includegraphics [
            width = 0.48\textwidth,
            % angle = -90,
        ] {pics/rough_frame_01.jpg}
        &
        \includegraphics [
            width = 0.48\textwidth,
            % angle = -90,
        ] {pics/rough_frame_02.jpg}

        \\
        &
        \\

        \captionof{figure} {Rough sketches of core block frame.}
        \label{fig:cbFrameRoughOne}
        &
        \captionof{figure} {Rough sketches of core block frame.}
        \label{fig:cbFrameRoughTwo}

        \\

    \end{tabularx}
\end{center}

\subsection{Thursday: Dec 04 2025}

\subsubsection{Printed Frame and Mounts}

Printed frame will have a base and necessary mounts for motor and limit switches.
The mounts are attachable to the base frame. See figure \ref{fig:cbFrameRoughThree}
for some more rough sketches of the frame.

\begin{figure}
    \centering
    \includegraphics [
        width = 0.5\textwidth,
    ] {pics/rough_frame_03.jpg}
    \captionof{figure} {Rough sketches of the frame.}
    \label{fig:cbFrameRoughThree}
\end{figure}

\subsubsection{Attaching Mechanisms}

Figures \ref{fig:snapInV1} and \ref{fig:snapInV2} show two designs for the snap in mechanism.

\begin{center}
    \begin{tabularx} {\textwidth} {
            >{\centering \arraybackslash}m{0.49\textwidth}
            >{\centering \arraybackslash}m{0.49\textwidth}
        }

        \includegraphics [
            width = 0.48\textwidth,
            % angle = -90,
        ] {pics/snap_in_v1.png}
        &
        \includegraphics [
            width = 0.48\textwidth,
            % angle = -90,
        ] {pics/snap_in_v2.png}

        \\
        &
        \\

        \captionof{figure} {CAD design of snap in version 1.}
        \label{fig:snapInV1}
        &
        \captionof{figure} {CAD design of snap in version 2.}
        \label{fig:snapInV2}

        \\

    \end{tabularx}
\end{center}

Figure \ref{fig:testPrints} shows the result of test prints of snap-in v1 and v2.
From the result, we've decided to pick version 1 with 0.2 mm print clearance.

\begin{figure}
    \centering
    \includegraphics [
        width = 0.5\textwidth,
    ] {pics/test_prints.jpg}
    \captionof{figure} {Result of test prints of snap in mechanism.}
    \label{fig:testPrints}
\end{figure}


\end{document}
