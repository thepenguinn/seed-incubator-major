\documentclass[../weekly]{subfiles}

\begin{document}

\letDaniel

This week, design for the panels and frames for the core block almost reached its
completion. Test printed several snap fit joints. Ordered motors and limit switches for
the core block.

\subsection{Frames for Core Block}

Printed frame CAD design is reaching its completion. There are 3 types of frames:
mid frame (3 frames), bottom frame (2 frames), and side frame (2 frames). The foam
part of these frame will be different. But the design for printed frames are similar
for these three types. The only difference is the frame length and frame height.

The design of mid frame is shown in figures \ref{fig:foamFrame} to \ref{fig:rack}. They
are parametrically modelled, so the bottom and side frame designs can be automatically
generated from the same.

\begin{center}
    \begin{tabularx} {\textwidth} {
            >{\centering \arraybackslash}m{0.49\textwidth}
            >{\centering \arraybackslash}m{0.49\textwidth}
        }

        \includegraphics [
            width = 0.48\textwidth,
            % angle = -90,
        ] {pics/foam_frame.png}
        &
        \includegraphics [
            width = 0.48\textwidth,
            % angle = -90,
        ] {pics/printed_frame.png}

        \\
        &
        \\

        \captionof{figure} {CAD design of the foam mid panel frame.}
        \label{fig:foamFrame}
        &
        \captionof{figure} {CAD design of printed mid panel frame.}
        \label{fig:printedFrame}

        \\

    \end{tabularx}
\end{center}

\begin{center}
    \begin{tabularx} {\textwidth} {
            >{\centering \arraybackslash}m{0.49\textwidth}
            >{\centering \arraybackslash}m{0.49\textwidth}
        }

        \includegraphics [
            width = 0.48\textwidth,
            % angle = -90,
        ] {pics/motor_mount.png}
        &
        \includegraphics [
            width = 0.48\textwidth,
            % angle = -90,
        ] {pics/limit_switch_mount.png}

        \\
        &
        \\

        \captionof{figure} {CAD design of motor mount.}
        \label{fig:MotorMount}
        &
        \captionof{figure} {CAD design of limit switch mount.}
        \label{fig:limitSwitchMount}

        \\

    \end{tabularx}
\end{center}

\begin{center}
    \begin{tabularx} {\textwidth} {
            >{\centering \arraybackslash}m{0.49\textwidth}
            >{\centering \arraybackslash}m{0.49\textwidth}
        }

        \includegraphics [
            width = 0.48\textwidth,
            % angle = -90,
        ] {pics/foam_connector.png}
        &
        \includegraphics [
            width = 0.48\textwidth,
            % angle = -90,
        ] {pics/rack.png}

        \\
        &
        \\

        \captionof{figure} {CAD design of foam and printed frame connector.}
        \label{fig:foamConnector}
        &
        \captionof{figure} {CAD design of rack.}
        \label{fig:rack}

        \\

    \end{tabularx}
\end{center}

\subsection{Test Prints for Snap Fit Connectors}

Tested several revisions of the snap fit joints. None of them works properly. Need
to improvise.

\begin{figure}
    \centering
    \includegraphics [
        width = 0.5\textwidth,
    ] {pics/test_prints.jpg}
    \captionof{figure} {Test prints for snap fit joints.}
    \label{fig:testPrint}
\end{figure}

\subsection{N20 Motors and Limit Switches}

Purchased motors and limit switches for the panels. Along with the gear belt
for GMU rail.

\begin{figure}
    \centering
    \includegraphics [
        width = 0.5\textwidth,
    ] {pics/components.jpg}
    \captionof{figure} {N20 motors and limit switches for the core block panels.}
    \label{fig:components}
\end{figure}

\end{document}
