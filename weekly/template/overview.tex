\def\calSide{0.12\textwidth}
\def\calSpace{0.02\textwidth}
\def\pillLength{0.017\textwidth}
\def\pillHeight{0.8cm}
\def\pillRadius{2pt}

\def\tblRadius{7pt}
\def\tblCellPaddedHeight{0.100\textwidth}
\def\tblCellWidth{0.194\textwidth}
\def\tblSpace{0.014\textwidth}
\def\tblCellPad{0.4cm}

\def\tblTopPad{10pt}

% \pgfmathsetmacro{\tblCellHeight}{\tblCellPaddedHeight - \tblCellPad}

\newcommand\tblDrawTop[0] {

    \draw
    (origin) coordinate (tmp)
    (tmp) ++(\calSide + \tblSpace, 0) coordinate (tmp)
    ;

    \footnotesize

    \foreach \pName/\pColor in {
        ANJANA/colorP1Dark,
        ASWATHEERTHA/colorP2Dark,
        ATHIRA/colorP3Dark,
        DANIEL/colorP4Dark%
    } {
        \draw

        (tmp) coordinate (start)
        (start) ++(\tblCellWidth, 0) coordinate (end)
        ++(\tblSpace, 0) coordinate (tmp)

        ($(start)!0.5!(end)$)
        node (tmpNameNode) [
            inner sep = 0pt,
            outer sep = 0pt,
            above = \tblTopPad,
            anchor = south,
        ] {\color{\pColor}\texttt{\pName}}
        ;
    }

}

% text width color
\newcommand\tblDrawCellContent[3] {
    \begin{minipage} {#2}
        \color{#3}
        \centering
        \scriptsize
        \ttfamily
        #1
    \end{minipage}
}

\newcommand\tblDrawTestCellHeight[4] {
    \begin{scope}
        \clip
        (0, 0)
        node (tmpTestCellHeightNode) [anchor = north west, inner sep = 0pt, outer sep = 0pt] {
            \pgfmathparse{\tblCellWidth - \tblCellPad}
            \begin{tabular} {
                    cccc
                }
                \tblDrawCellContent{#1}{\pgfmathresult pt}{colorP4Dark} &
                \tblDrawCellContent{#2}{\pgfmathresult pt}{colorP4Dark} &
                \tblDrawCellContent{#3}{\pgfmathresult pt}{colorP4Dark} &
                \tblDrawCellContent{#4}{\pgfmathresult pt}{colorP4Dark}
                \\
            \end{tabular}
        }
        ;
        \draw
        (tmpTestCellHeightNode.north) coordinate (A)
        (tmpTestCellHeightNode.south) coordinate (B)
        ;
    \end{scope}
    \calcLength(A,B){tblCellTestHeight}
}

%% <name> <corner> <oppCorner>
\newcommand\innerEnclosureNode[4] {
    \filldraw [
        #4,
        rounded corners = \tblRadius,
        fill = #4,
        % draw = black,
    ]
    (#2) rectangle (#3)
    ;
    \draw
    let \p{TEMPDIMA} = ($(#2)-(#3)$),
    \n{TEMPDIMX} = {veclen(\x{TEMPDIMA}, 0)},
    \n{TEMPDIMY} = {veclen(0, \y{TEMPDIMA})}
    in
    ($(#2)!0.50!(#3)$) ++(4pt, 0)
    node (tmpNode) [
        % black,
        % rounded corners = \tblRadius,
        % fill = #4,
        % draw = black,
        inner sep = 0pt,
        outer sep = 0pt,
        minimum height = \n{TEMPDIMY},
        minimum width = \n{TEMPDIMX},
        anchor = center,
        align = center,
    ] {
        \hfill
        \pgfmathparse{\n{TEMPDIMX} - \tblCellPad}
        \tblDrawCellContent {#1} {\pgfmathresult pt} {#4Dark}
    }
    ;
}

% date dateColor p1Text p2Text p3Text p4Text
\newcommand\tblDrawRow[6] {

    \tblDrawTestCellHeight{#3} {#4} {#5} {#6}

    \pgfmathsetmacro{\tblCellMaxHeight}{max(\tblCellTestHeight pt, \tblCellHeight)}

    \fill [
        #2,
        very thick,
        rounded corners = \tblRadius,
    ]

    (itmp) coordinate (tmp) coordinate (start)
    (start) ++(\calSide, -\tblCellMaxHeight pt - \tblCellPad) coordinate (end)
    (start) rectangle (end)
    ++(\tblSpace, {\tblCellMaxHeight pt + \tblCellPad}) coordinate (tmp)

    (end) ++(-\calSide, -\tblSpace) coordinate (itmp)

    ;

    \draw [
        white,
    ]
    ($(start)!0.5!(end)$)
    node [
        font = \Large,
        anchor = center,
    ]
    {\texttt{#1}}
    ;

    \tblDrawCell {#3} {colorP1}
    \tblDrawCell {#4} {colorP2}
    \tblDrawCell {#5} {colorP3}
    \tblDrawCell {#6} {colorP4}

}

% text fgcol
\newcommand\tblDrawCell[2] {

    \fill [
        very thick,
        rounded corners = 14pt,
    ]

    (tmp) coordinate (start)
    (start) ++(\tblCellWidth, -\tblCellMaxHeight pt - \tblCellPad) coordinate (end)
    ++(\tblSpace, {\tblCellMaxHeight pt + \tblCellPad}) coordinate (tmp)

    ;

    \innerEnclosureNode{#1}{start}{end}{#2}

}

% date day fg bg dayfg p1 p2 p3 p4
\newcommand\calDrawCell[9] {
    \fill [
        #4,
        very thick,
        rounded corners = 14pt,
    ]

    (tmp) coordinate (start)
    (start) ++(\calSide, -\calSide) coordinate (end)
    (start) rectangle (end)
    ++(\calSpace, \calSide) coordinate (tmp)

    ;

    \draw [
        #3,
    ]
    ($(start)!0.5!(end)$)
    node [
        font = \Large,
        anchor = center,
    ]
    {\texttt{#1}}
    ;

    \draw [
        #5,
    ]
    ($(start)!0.5!(end)$) ++(0, \calSide / 2)
    node [
        font = \large,
        anchor = south,
        above = 6pt,
    ]
    {\texttt{#2}}
    ;

    \draw (start |- end) coordinate (botLeft) ++(0, -10pt) coordinate (pillHori) ;

    \draw ($(botLeft)!0.2!(end)$) coordinate (p1Vert);
    \draw ($(botLeft)!0.4!(end)$) coordinate (p2Vert);
    \draw ($(botLeft)!0.6!(end)$) coordinate (p3Vert);
    \draw ($(botLeft)!0.8!(end)$) coordinate (p4Vert);

    \ifthenelse {#6 = 1} {
        \fill [
            colorP1,
            rounded corners = \pillRadius,
        ]
        (pillHori -| p1Vert) ++(-\pillLength / 2, 0) rectangle ++(\pillLength, -\pillHeight)
        ;
    }{}

    \ifthenelse {#7 = 1} {
        \fill [
            colorP2,
            rounded corners = \pillRadius,
        ]
        (pillHori -| p2Vert) ++(-\pillLength / 2, 0) rectangle ++(\pillLength, -\pillHeight)
        ;
    }{}

    \ifthenelse {#8 = 1} {
        \fill [
            colorP3,
            rounded corners = \pillRadius,
        ]
        (pillHori -| p3Vert) ++(-\pillLength / 2, 0) rectangle ++(\pillLength, -\pillHeight)
        ;
    }{}

    \ifthenelse {#9 = 1} {
        \fill [
            colorP4,
            rounded corners = \pillRadius,
        ]
        (pillHori -| p4Vert) ++(-\pillLength / 2, 0) rectangle ++(\pillLength, -\pillHeight)
        ;
    }{}

}

\newcommand\fillTblCellHeight[0] {
    \draw
    (tmpNameNode.north |- origin) coordinate (end)
    (tmpNameNode.north) coordinate (start)
    ;

    \calcLength(start,end) {tblNameHeight}

    \pgfmathsetmacro{\tmp}{\spaceLeft - \tblNameHeight - (7 * \tblSpace) - (7 * \tblCellPad)}
    \xdef\spaceLeft{\tmp}

    \pgfmathsetmacro{\tblCellIdealHeight}{\spaceLeft / 7}

    \xdef\dayCount{7}

    \foreach \date/\holiday/\pOne/\pTwo/\pThree/\pFour in \tblData {
        \tblDrawTestCellHeight{\pOne} {\pTwo} {\pThree} {\pFour}

        \ifthenelse{\lengthtest{\tblCellTestHeight pt > \tblCellIdealHeight pt}} {

            \pgfmathsetmacro{\tmp}{\spaceLeft - \tblCellTestHeight}
            \xdef\spaceLeft{\tmp}
            \pgfmathsetmacro{\tmp}{\dayCount - 1}
            \xdef\dayCount{\tmp}

            \pgfmathsetmacro{\tmp}{\spaceLeft / \dayCount}
            \xdef\tblCellIdealHeight{\tmp}

        } {
        }
    }

    \pgfmathsetmacro{\tmp}{\spaceLeft / \dayCount}
    \xdef\tblCellHeight{\tmp}
}


\def\week{03}
\def\weekRange{OCT 05 -- OCT 11}

\newcommand\calData{
    05/SUN/1/1/1/1/1,
    06/MON/0/1/0/1/0,
    07/TUE/0/1/1/1/1,
    08/WED/0/1/0/1/1,
    09/THU/0/1/1/1/1,
    10/FRI/0/0/1/0/1,
    11/SAT/0/1/1/1/1
}

\newcommand\tblData{
    05/1/Somecontent that is really really really long/1/1/1,
    06/0/1/0/1/0,
    07/0/1/1/1/1,
    08/0/1/0/1/1,
    09/0/1/1/1/1,
    10/0/0/1/0/1,
    11/0/1/1/1/1
}


\chapter{Seed Incubation Plant: Weekly Report}
\vspace{-0.6cm}

\begin{center}

    \begin{tikzpicture} [
        ]

        \draw (0, 0) coordinate (tmp) ;

        \foreach \date/\day/\holiday/\pOne/\pTwo/\pThree/\pFour in \calData {
            \ifthenelse{\holiday = 1} {
                \calDrawCell{\date}{\day}{white}{colorSun}{colorSun} {\pOne} {\pTwo} {\pThree} {\pFour}
            } {
                \calDrawCell{\date}{\day}{white}{colorNonSun}{colorNonSunDark} {\pOne} {\pTwo} {\pThree} {\pFour}
            }
        }

    \end{tikzpicture}

    \vspace{0.6cm}

    \newdimen\tblHeight
    \tblHeight=\dimexpr\textheight-\pagetotal\relax

    \def\spaceLeft{\the\tblHeight}
    \pgfmathsetmacro{\tmp}{\spaceLeft - 3pt}
    \xdef\spaceLeft{\tmp}

    \begin{tikzpicture} [
        ]

        \draw (0, 0) coordinate (origin) coordinate (itmp) ;

        \tblDrawTop

        \fillTblCellHeight

        \foreach \date/\holiday/\pOne/\pTwo/\pThree/\pFour in \tblData {
            \ifthenelse{\holiday = 1} {
                \tblDrawRow {\date} {colorSun} {\pOne} {\pTwo} {\pThree} {\pFour}
            } {
                \tblDrawRow {\date} {colorNonSun} {\pOne} {\pTwo} {\pThree} {\pFour}
            }
        }

    \end{tikzpicture}

    \vfill

\end{center}
