\usepackage[margin = 0.7in]{geometry}
\usepackage{tikz}
\usepackage{titlesec}
\usepackage{siunitx}
\usepackage[table]{xcolor}
\usepackage{ifthen}
\usepackage{nicematrix}
\usepackage{booktabs}
\usepackage{tabularx}
\usepackage{fp}
\usepackage{graphicx}
\usepackage{fancyhdr}
\usepackage{setspace}
\usepackage{hyperref}
\usepackage{caption}
\usepackage{amsmath}
\usepackage{float}
\usepackage{tcolorbox}
\usepackage{subfiles}

\usepackage{blindtext}

\usetikzlibrary{calc}
\usetikzlibrary{positioning}
\usetikzlibrary{math}

\makeatletter
\def\calcLength(#1,#2)#3{%
\pgfpointdiff{\pgfpointanchor{#1}{center}}%
             {\pgfpointanchor{#2}{center}}%
\pgf@xa=\pgf@x%
\pgf@ya=\pgf@y%
\FPeval\@temp@a{\pgfmath@tonumber{\pgf@xa}}%
\FPeval\@temp@b{\pgfmath@tonumber{\pgf@ya}}%
\FPeval\@temp@sum{(\@temp@a*\@temp@a+\@temp@b*\@temp@b)}%
\FProot{\FPMathLen}{\@temp@sum}{2}%
\FPround\FPMathLen\FPMathLen5\relax
\global\expandafter\edef\csname #3\endcsname{\FPMathLen}
}
\makeatother

\renewcommand{\tabularxcolumn}[1]{m{#1}}

\definecolor{colorRed} {HTML} {E44E54}
\colorlet{colorLightRed}{colorRed!70!white}

\colorlet{colorPill}{black!30!white}

\definecolor{colorPastelGreen} {HTML} {97B3AE}

\colorlet{colorNonSun}{white!15!colorPastelGreen}
\colorlet{colorNonSunDark}{black!30!colorNonSun}

\colorlet{colorSun}{colorLightRed}
\colorlet{colorSunDark}{black!30!colorSun}

\definecolor{colorP1} {HTML} {D2E0D3}
\definecolor{colorP2} {HTML} {F0DDD6}
\definecolor{colorP3} {HTML} {F2C3B9}
\definecolor{colorP4} {HTML} {D6CBBF}

\colorlet{colorP1Dark}{black!60!colorP1}
\colorlet{colorP2Dark}{black!60!colorP2}
\colorlet{colorP3Dark}{black!60!colorP3}
\colorlet{colorP4Dark}{black!60!colorP4}

\colorlet{colorP1LightDark}{black!10!colorP1}
\colorlet{colorP2LightDark}{black!10!colorP2}
\colorlet{colorP3LightDark}{black!10!colorP3}
\colorlet{colorP4LightDark}{black!10!colorP4}

\titleformat{\chapter} [display]
{\bfseries\normalfont\huge\filright\sffamily\vspace{-2cm}}
{\color{colorNonSunDark} \Large\texttt{{\week} \hfill \weekRange} \vspace{1em}}
{1pc}
{\color{colorNonSunDark}\titlerule\vspace{1.0em}\scshape\centering\color{colorNonSunDark}}
[\color{colorNonSunDark}\vspace{0.65em}{\titlerule[1pt]}]

\titleformat{\section}
{\vspace{-0.35em}\bfseries\normalfont\Large\filright\sffamily}
{\thesection}
{8pt}
{\scshape}
{}

\titleformat{\subsection}
{\vspace{-0.35em}\bfseries\normalfont\filright\sffamily\large}
% {}
% {0pt}
{\thesubsection}
{8pt}
{\scshape}
{}

%%

% alerts

\colorlet{colorAccent}{colorP1}
\colorlet{colorAccentDark}{colorP1Dark}

\newcommand\alertCaution[1] {
    \begin{tcolorbox}[
            coltitle = colorAccentDark,
            colbacktitle = colorAccent,
            colback = colorAccent,
            colframe = colorAccent,
            title=Caution,
            fonttitle=\bfseries,
            detach title,
        ]
        \color{colorAccentDark}
        \begin{minipage}[t]{0.18\textwidth}
            \begin{flushleft}
                \tcbtitle
            \end{flushleft}
        \end{minipage}
        \begin{minipage}[t]{0.8\textwidth}
            #1
        \end{minipage}
    \end{tcolorbox}
}

\newcommand\alertWarning[1] {
    \begin{tcolorbox}[
            coltitle = colorAccentDark,
            colbacktitle = colorAccent,
            colback = colorAccent,
            colframe = colorAccent,
            title=Warning,
            fonttitle=\bfseries,
            detach title,
        ]
        \color{colorAccentDark}
        \begin{minipage}[t]{0.18\textwidth}
            \begin{flushleft}
                \tcbtitle
            \end{flushleft}
        \end{minipage}
        \begin{minipage}[t]{0.8\textwidth}
            #1
        \end{minipage}
    \end{tcolorbox}
}

\newcommand\alertImportant[1] {
    \begin{tcolorbox}[
            coltitle = colorAccentDark,
            colbacktitle = colorAccent,
            colback = colorAccent,
            colframe = colorAccent,
            title=Important,
            fonttitle=\bfseries,
            detach title,
        ]
        \color{colorAccentDark}
        \begin{minipage}[t]{0.18\textwidth}
            \begin{flushleft}
                \tcbtitle
            \end{flushleft}
        \end{minipage}
        \begin{minipage}[t]{0.8\textwidth}
            #1
        \end{minipage}
    \end{tcolorbox}
}

\newcommand\alertTip[1] {
    \begin{tcolorbox}[
            coltitle = colorAccentDark,
            colbacktitle = colorAccent,
            colback = colorAccent,
            colframe = colorAccent,
            title=Tip,
            fonttitle=\bfseries,
            detach title,
        ]
        \color{colorAccentDark}
        \begin{minipage}[t]{0.18\textwidth}
            \begin{flushleft}
                \tcbtitle
            \end{flushleft}
        \end{minipage}
        \begin{minipage}[t]{0.8\textwidth}
            #1
        \end{minipage}
    \end{tcolorbox}
}

\newcommand\alertNote[1] {
    \begin{tcolorbox}[
            coltitle = colorAccentDark,
            colbacktitle = colorAccent,
            colback = colorAccent,
            colframe = colorAccent,
            title=Note,
            fonttitle=\bfseries,
            detach title,
        ]
        \color{colorAccentDark}
        \begin{minipage}[t]{0.18\textwidth}
            \begin{flushleft}
                \tcbtitle
            \end{flushleft}
        \end{minipage}
        \begin{minipage}[t]{0.8\textwidth}
            #1
        \end{minipage}
    \end{tcolorbox}
}

%% let

\newcommand\createSection[2] {
    \newdimen\tmpHeight
    \tmpHeight=\dimexpr\textheight-\pagetotal\relax

    \def\spaceLeft{\the\tmpHeight}
    \pgfmathsetmacro{\tmp}{\spaceLeft - 0pt}
    \xdef\spaceLeft{\tmp}

    \ifthenelse{\lengthtest{\spaceLeft pt < 75pt}}
    {\pagebreak}
    {}

    \colorlet{colorAccent}{colorP#1}
    \colorlet{colorAccentDark}{colorP#1Dark}
    \color{colorP#1Dark}
    \section{#2}
    \vspace{-0.85cm}
    \begin{center}
        \color{colorP#1}
        \begin{tikzpicture}
            \filldraw [
                colorP#1,
                rounded corners = 0.5pt,
            ]
            (0, 0) rectangle ++(\textwidth, -1.25pt)
            ;
        \end{tikzpicture}
    \end{center}
}

\newcommand\letAnjana[0] {
    \createSection{1}{Anjana}
}

\newcommand\letAswatheertha[0] {
    \createSection{2}{Aswatheertha}
}

\newcommand\letAthira[0] {
    \createSection{3}{Athira}
}

\newcommand\letDaniel[0] {
    \createSection{4}{Daniel}
}
