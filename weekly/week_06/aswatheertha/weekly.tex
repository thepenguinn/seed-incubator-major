\documentclass[../weekly]{subfiles}

\begin{document}
\letAswatheertha

\noindent This week's focus was on researching and planning the rail-based motion system for plant imaging. The study mainly involved stepper motor control, driver selection, and microcontroller interfacing. A detailed wiring structure was reviewed and drafted for implementation.

\subsection{Rail System Circuit Research}
\begin{enumerate}
    \item \textbf{Pin expansion setup:}
    \begin{itemize}
        \item Decided to use \textbf{ATmega328P} (Nano form) to control the rail movement.
        \item Considered alternatives like \textbf{I2C expanders} and \textbf{shift registers}, but the Nano was preferred because it:
        \begin{itemize}
            \item Provides direct hardware control of stepper pulses (better timing precision).
            \item I2C only gives simple on/off mechanisms like for Leds.
            \item Gives enough GPIO pins for step, direction, limit switches, LEDs, and serial communication- total 14 pins in atmega.
        \end{itemize}
    \end{itemize}
    
    \begin{figure}[H]
    \centering
    \includegraphics[width=0.75\textwidth]{pics/atmegapinout.png}
    \caption{ATmega328P Pinout Reference}
    \end{figure}

    \item \textbf{Wiring Setup Draft:}
    \begin{itemize}
        \item Two DRV8825 drivers are used for X and Y stepper motors.
        \item Each driver receives STEP and DIR signals from the Atmega controller.
        \item Limit switches are connected to the Atmega controller digital input pins with internal/ external pull-ups.
        \item LEDs for lighting are driven through BJT switching to handle higher current.
        \item Serial communication (UART:Most probably) is established between Atmega controller and ESP32-CAM for sending movement or capture commands.
    \end{itemize}
    
    \begin{figure}[h!]
\centering
\begin{minipage}[b]{0.48\textwidth}
    \centering
    \includegraphics[width=\textwidth]{pics/notes1.jpg}
    \caption*{(a)}
\end{minipage}
\hfill
\begin{minipage}[b]{0.48\textwidth}
    \centering
    \includegraphics[width=\textwidth]{pics/notes2.jpg}
    \caption*{(b)}
\end{minipage}

\vspace{8pt}

\begin{minipage}[b]{0.48\textwidth}
    \centering
    \includegraphics[width=\textwidth]{pics/notes3.jpg}
    \caption*{(c)}
\end{minipage}
\hfill
\begin{minipage}[b]{0.48\textwidth}
    \centering
    \includegraphics[width=\textwidth]{pics/notes4.jpg}
    \caption*{(d)}
\end{minipage}

\caption{ notes and wiring sketches for rail system design.}
\end{figure}


    \item \textbf{Connection Summary:}
    \begin{itemize}
        \item \textbf{DRV8825(each):} STEP $\rightarrow$ D2(X) \&  D5(Y), DIR $\rightarrow$ D3(X) \&  D6(Y), SLEEP + RESET $\rightarrow$ 10kohm → 5V(pin 7), VMOT $\rightarrow$ 12V (power) with 100 $\mu$ F capacitor.
        \item \textbf{Limit Switches:} D8 (X), D9 (Y) $\rightarrow$ Switch $\rightarrow$ GND, with pull-up enabled.
        \item \textbf{LEDs:} Nano digital pin $\rightarrow$ Gate/Base $\rightarrow$ Transistor $\rightarrow$ LED $\rightarrow$ 12V.
        \item \textbf{Serial:} Nano TX/RX(pin 2,3) $\leftrightarrow$  ESP32-CAM RX/TX, common GND required.
        \item \textbf{Power:} 12V supply for motors and LEDs; 5V regulated for Nano + ESP32.
    \end{itemize}


\end{enumerate}

\end{document}
