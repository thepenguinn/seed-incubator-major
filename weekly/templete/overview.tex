\documentclass[11pt, a4paper]{report}

\usepackage[margin = 0.7in]{geometry}
\usepackage{tikz}
\usepackage{titlesec}
\usepackage{siunitx}
\usepackage[table]{xcolor}
\usepackage{ifthen}
\usepackage{nicematrix}
\usepackage{booktabs}
\usepackage{tabularx}

\usetikzlibrary{calc}
\usetikzlibrary{positioning}

\renewcommand{\tabularxcolumn}[1]{m{#1}}

\definecolor{colorRed} {HTML} {E44E54}
\colorlet{colorLightRed}{colorRed!70!white}

\colorlet{colorPill}{black!30!white}

\definecolor{colorPastelGreen} {HTML} {97B3AE}

\colorlet{colorNonSun}{white!15!colorPastelGreen}
\colorlet{colorNonSunDark}{black!30!colorNonSun}

\colorlet{colorSun}{colorLightRed}
\colorlet{colorSunDark}{black!30!colorSun}

\definecolor{colorP1} {HTML} {D2E0D3}
\definecolor{colorP2} {HTML} {F0DDD6}
\definecolor{colorP3} {HTML} {F2C3B9}
\definecolor{colorP4} {HTML} {D6CBBF}

\colorlet{colorP1Dark}{black!60!colorP1}
\colorlet{colorP2Dark}{black!60!colorP2}
\colorlet{colorP3Dark}{black!60!colorP3}
\colorlet{colorP4Dark}{black!60!colorP4}

\titleformat{\chapter} [display]
{\bfseries\normalfont\huge\filright\sffamily\vspace{-2cm}}
{\color{colorNonSunDark} \Large\texttt{WEEK {01} \hfill OCT 05 -- OCT 11} \vspace{1em}}
{1pc}
{\color{colorNonSunDark}\titlerule\vspace{1.0em}\scshape\centering\color{colorNonSunDark}}
[\color{colorNonSunDark}\vspace{0.65em}{\titlerule[1pt]}]

\titleformat{\section}
{\vspace{-0.35em}\bfseries\normalfont\large\filright\sffamily}
{\thesection}
{8pt}
{\scshape}
{}

\titleformat{\subsection}
{\vspace{-0.35em}\bfseries\normalfont\filright\sffamily}
% {}
% {0pt}
{\thesubsection}
{8pt}
{\scshape}
{}


\begin{document}

\def\calSide{0.12\textwidth}
\def\calSpace{0.02\textwidth}
\def\pillLength{0.017\textwidth}
\def\pillHeight{0.8cm}
\def\pillRadius{2pt}

\def\tblRadius{7pt}
\def\tblCellPaddedHeight{0.100\textwidth}
\def\tblCellWidth{0.194\textwidth}
\def\tblSpace{0.014\textwidth}
\def\tblCellPad{0.4cm}

\pgfmathsetmacro{\tblCellHeight}{\tblCellPaddedHeight - \tblCellPad}

% text width color
\newcommand\tblDrawCellContent[3] {
    \begin{minipage} {#2}
        \color{#3}
        \centering
        \scriptsize
        \ttfamily
        #1
    \end{minipage}
}

\newcommand\tblDrawTestCellHeight[4] {
    \begin{scope}
        \clip
        (0, 0)
        node (tmpTestCellHeightNode) [anchor = north west, inner sep = 0pt, outer sep = 0pt] {
            \pgfmathparse{\tblCellWidth - \tblCellPad}
            \begin{tabular} {
                    cccc
                }
                \tblDrawCellContent{#1}{\pgfmathresult pt}{colorP4Dark} &
                \tblDrawCellContent{#2}{\pgfmathresult pt}{colorP4Dark} &
                \tblDrawCellContent{#3}{\pgfmathresult pt}{colorP4Dark} &
                \tblDrawCellContent{#4}{\pgfmathresult pt}{colorP4Dark}
                \\
            \end{tabular}
        }
        ;
        \draw
        (tmpTestCellHeightNode.north) coordinate (A)
        (tmpTestCellHeightNode.south) coordinate (B)
        ;
    \end{scope}
    \calcLength(A,B){tblCellTestHeight}
}

%% <name> <corner> <oppCorner>
\newcommand\innerEnclosureNode[4] {
    \filldraw [
        #4,
        rounded corners = \tblRadius,
        fill = #4,
        % draw = black,
    ]
    (#2) rectangle (#3)
    ;
    \draw
    let \p{TEMPDIMA} = ($(#2)-(#3)$),
    \n{TEMPDIMX} = {veclen(\x{TEMPDIMA}, 0)},
    \n{TEMPDIMY} = {veclen(0, \y{TEMPDIMA})}
    in
    ($(#2)!0.50!(#3)$) ++(4pt, 0)
    node (tmpNode) [
        % black,
        % rounded corners = \tblRadius,
        % fill = #4,
        % draw = black,
        inner sep = 0pt,
        outer sep = 0pt,
        minimum height = \n{TEMPDIMY},
        minimum width = \n{TEMPDIMX},
        anchor = center,
        align = center,
    ] {
        \hfill
        \pgfmathparse{\n{TEMPDIMX} - \tblCellPad}
        \tblDrawCellContent {#1} {\pgfmathresult pt} {#4Dark}
    }
    ;
}

% date dateColor p1Text p2Text p3Text p4Text
\newcommand\tblDrawRow[6] {

    \tblDrawTestCellHeight{#3} {#4} {#5} {#6}

    \pgfmathsetmacro{\tblCellMaxHeight}{max(\tblCellTestHeight pt, \tblCellHeight)}

    \fill [
        #2,
        very thick,
        rounded corners = \tblRadius,
    ]

    (itmp) coordinate (tmp) coordinate (start)
    (start) ++(\calSide, -\tblCellMaxHeight pt - \tblCellPad) coordinate (end)
    (start) rectangle (end)
    ++(\tblSpace, {\tblCellMaxHeight pt + \tblCellPad}) coordinate (tmp)

    (end) ++(-\calSide, -\tblSpace) coordinate (itmp)

    ;

    \draw [
        white,
    ]
    ($(start)!0.5!(end)$)
    node [
        font = \Large,
        anchor = center,
    ]
    {\texttt{#1}}
    ;

    \tblDrawCell {#3} {colorP1}
    \tblDrawCell {#4} {colorP2}
    \tblDrawCell {#5} {colorP3}
    \tblDrawCell {#6} {colorP4}
}

% text fgcol
\newcommand\tblDrawCell[2] {

    \fill [
        colorP1,
        very thick,
        rounded corners = 14pt,
    ]

    (tmp) coordinate (start)
    (start) ++(\tblCellWidth, -\tblCellMaxHeight pt - \tblCellPad) coordinate (end)
    % (start) rectangle (end)
    ++(\tblSpace, {\tblCellMaxHeight pt + \tblCellPad}) coordinate (tmp)

    ;

    \innerEnclosureNode{#1}{start}{end}{#2}

}

% date day fg bg dayfg p1 p2 p3 p4
\newcommand\calDrawCell[9] {
    \fill [
        #4,
        very thick,
        rounded corners = 14pt,
    ]

    (tmp) coordinate (start)
    (start) ++(\calSide, -\calSide) coordinate (end)
    (start) rectangle (end)
    ++(\calSpace, \calSide) coordinate (tmp)

    ;

    \draw [
        #3,
    ]
    ($(start)!0.5!(end)$)
    node [
        font = \Large,
        anchor = center,
    ]
    {\texttt{#1}}
    ;

    \draw [
        #5,
    ]
    ($(start)!0.5!(end)$) ++(0, \calSide / 2)
    node [
        font = \large,
        anchor = south,
        above = 6pt,
    ]
    {\texttt{#2}}
    ;

    \draw (start |- end) coordinate (botLeft) ++(0, -10pt) coordinate (pillHori) ;

    \draw ($(botLeft)!0.2!(end)$) coordinate (p1Vert);
    \draw ($(botLeft)!0.4!(end)$) coordinate (p2Vert);
    \draw ($(botLeft)!0.6!(end)$) coordinate (p3Vert);
    \draw ($(botLeft)!0.8!(end)$) coordinate (p4Vert);

    \ifthenelse {#6 = 1} {
        \fill [
            colorP1,
            rounded corners = \pillRadius,
        ]
        (pillHori -| p1Vert) ++(-\pillLength / 2, 0) rectangle ++(\pillLength, -\pillHeight)
        ;
    }{}

    \ifthenelse {#7 = 1} {
        \fill [
            colorP2,
            rounded corners = \pillRadius,
        ]
        (pillHori -| p2Vert) ++(-\pillLength / 2, 0) rectangle ++(\pillLength, -\pillHeight)
        ;
    }{}

    \ifthenelse {#8 = 1} {
        \fill [
            colorP3,
            rounded corners = \pillRadius,
        ]
        (pillHori -| p3Vert) ++(-\pillLength / 2, 0) rectangle ++(\pillLength, -\pillHeight)
        ;
    }{}

    \ifthenelse {#9 = 1} {
        \fill [
            colorP4,
            rounded corners = \pillRadius,
        ]
        (pillHori -| p4Vert) ++(-\pillLength / 2, 0) rectangle ++(\pillLength, -\pillHeight)
        ;
    }{}

}

\chapter{Seed Incubation Plant: Weekly Report}
\vspace{-0.6cm}

% {\color{colorPastelGreen!60!black} \Large\ttfamily WEEK 01 \hfill OCT 05 -- OCT 11}
% \vspace{1cm}
%
% \begin{tikzpicture} [
%         titleNode/.style = {
%             % ultra thick,
%             rectangle,
%             draw = white,
%             fill = colorSun,
%             rounded corners = 0.5cm,
%             inner sep = 0.75cm,
%             outer sep = 0cm,
%             % text width = 7.5cm,
%             text centered,
%         },
%     ]
%
%     \draw [white]
%     node (title) [titleNode, font = \LARGE] {\textsc{Seed Incubation Plant}}
%     ;
%
%     % \draw [colorNonSunDark]
%     % node (report) [font = \LARGE, right = 1.4cm of title] {\textsc{Weekly Report}}
%     % ;
%
% \end{tikzpicture}
% \hfill
% {\color{colorNonSunDark}\LARGE\textsc{Weekly Report}}
%
% \vspace{0.75cm}

\begin{center}

    \begin{tikzpicture} [
        ]

        \draw (0, 0) coordinate (tmp) ;

        \calDrawCell{05}{SUN}{white}{colorSun}{colorSunDark} {1} {1} {1} {1}

        \foreach \x/\y in {06/MON,07/TUE,08/WED,09/THU,10/FRI,11/SAT} {
            \calDrawCell{\x}{\y}{white}{colorNonSun}{colorNonSunDark} {1} {1} {1} {1}
        }

    \end{tikzpicture}

    \vspace{1cm}

    \begin{tikzpicture} [
        ]



        % \node at (3, 0) {\textValue} ;


        \draw (0, 0) coordinate (itmp) ;

        \tblDrawRow {05} {colorSun} {Really } {wowo} {wnop} {hai}
        \tblDrawRow {06} {colorNonSun} {Finished something that is really really important. Adding another sentence.} {wowo} {wnop} {hai}
        \tblDrawRow {07} {colorNonSun} {In the end they are all gone.} {wowo} {wnop} {hai}
        \tblDrawRow {08} {colorNonSun} {Really } {w s d s d s d s d s d d} {wnop} {Haven't done anything.}
        \tblDrawRow {09} {colorNonSun} {Really } {wowo} {wnop} {hai}
        \tblDrawRow {10} {colorNonSun} {Really } {wowo} {wnop} {hai}
        \tblDrawRow {11} {colorNonSun} {Really } {wowo} {wnop} {hai}

        % \tblDrawRow {colorNonSun}
        % \tblDrawRow {colorNonSun}
        % \tblDrawRow {colorNonSun}
        % \tblDrawRow {colorNonSun}
        % \tblDrawRow {colorNonSun}
        % \tblDrawRow {colorNonSun}

    \end{tikzpicture}



    % \newcommand\done[0]{
    %     Finish that thing this thing another thing.
    % }
    %
    % \small
    % \renewcommand\arraystretch{1.75}
    %
    % \begin{tabularx} {\textwidth} [] {
    %         % >{\columncolor{colorNonSun!40!white} \Large \centering \ttfamily \arraybackslash}m{\calSide}
    %         % >{\columncolor{colorP1} \ttfamily \color{colorP1Dark} \centering \arraybackslash}X
    %         % >{\columncolor{colorP2} \ttfamily \color{colorP2Dark} \centering \arraybackslash}X
    %         % >{\columncolor{colorP3} \ttfamily \color{colorP3Dark} \centering \arraybackslash}X
    %         % >{\columncolor{colorP4} \ttfamily \color{colorP4Dark} \centering \arraybackslash}X
    %         %%
    %         >{\Large \centering \ttfamily \arraybackslash \color{colorNonSunDark}}m{1cm}
    %         >{\ttfamily \color{colorP1Dark} \centering \arraybackslash}X
    %         >{\ttfamily \color{colorP2Dark} \centering \arraybackslash}X
    %         >{\ttfamily \color{colorP3Dark} \centering \arraybackslash}X
    %         >{\ttfamily \color{colorP4Dark} \centering \arraybackslash}X
    %     }
    %     \arrayrulecolor{colorNonSunDark}
    %     \toprule
    %
    %     & {\color{colorP1Dark} \ttfamily \centering Anjana} &
    %     {\centering Aswatheertha} &
    %     {\centering Athira} &
    %     {\centering Daniel} \\
    %
    %     \midrule
    %
    %     {\color{colorSun} 05} & \done & \done & \done & \done \\ \midrule
    %     {\color{colorNonSunDark} 06} & \done & \done & \done & \done \\ \midrule
    %     {\color{colorNonSunDark} 07} & \done & \done & \done & \done \\ \midrule
    %     {\color{colorNonSunDark} 08} & \done & \done & \done & \done \\ \midrule
    %     {\color{colorNonSunDark} 09} & \done & \done & \done & \done \\ \midrule
    %     {\color{colorNonSunDark} 10} & \done & \done & \done & \done \\ \midrule
    %     {\color{colorNonSunDark} 11} & \done & \done & \done & \done \\
    %
    %     \bottomrule
    %
    % \end{tabularx}


    % \begin{NiceTabularX}{\textwidth}{X X}
    %     \CodeBefore
    %     \arrayrulecolor{gray}
    %     \Body
    %     First Row Content & Another cell \\
    %     \Block{1-1}{This row will be taller.} & Second Row Content \\[2cm]
    %     Third Row Content & Final cell \\
    % \end{NiceTabularX}

    % \begin{tikzpicture} [
    %     ]
    %
    %
    % \end{tikzpicture}

\end{center}


\end{document}
