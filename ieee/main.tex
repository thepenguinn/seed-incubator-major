\documentclass[conference]{IEEEtran}

\usepackage[margin = 0.7in]{geometry}
\usepackage{tikz}
\usepackage{titlesec}
\usepackage{siunitx}
\usepackage[table]{xcolor}
\usepackage{ifthen}
\usepackage{nicematrix}
\usepackage{booktabs}
\usepackage{tabularx}

\usetikzlibrary{calc}
\usetikzlibrary{positioning}

\renewcommand{\tabularxcolumn}[1]{m{#1}}

\definecolor{colorRed} {HTML} {E44E54}
\colorlet{colorLightRed}{colorRed!70!white}

\colorlet{colorPill}{black!30!white}

\definecolor{colorPastelGreen} {HTML} {97B3AE}

\colorlet{colorNonSun}{white!15!colorPastelGreen}
\colorlet{colorNonSunDark}{black!30!colorNonSun}

\colorlet{colorSun}{colorLightRed}
\colorlet{colorSunDark}{black!30!colorSun}

\definecolor{colorP1} {HTML} {D2E0D3}
\definecolor{colorP2} {HTML} {F0DDD6}
\definecolor{colorP3} {HTML} {F2C3B9}
\definecolor{colorP4} {HTML} {D6CBBF}

\colorlet{colorP1Dark}{black!60!colorP1}
\colorlet{colorP2Dark}{black!60!colorP2}
\colorlet{colorP3Dark}{black!60!colorP3}
\colorlet{colorP4Dark}{black!60!colorP4}

\titleformat{\chapter} [display]
{\bfseries\normalfont\huge\filright\sffamily\vspace{-2cm}}
{\color{colorNonSunDark} \Large\texttt{WEEK {01} \hfill OCT 05 -- OCT 11} \vspace{1em}}
{1pc}
{\color{colorNonSunDark}\titlerule\vspace{1.0em}\scshape\centering\color{colorNonSunDark}}
[\color{colorNonSunDark}\vspace{0.65em}{\titlerule[1pt]}]

\titleformat{\section}
{\vspace{-0.35em}\bfseries\normalfont\large\filright\sffamily}
{\thesection}
{8pt}
{\scshape}
{}

\titleformat{\subsection}
{\vspace{-0.35em}\bfseries\normalfont\filright\sffamily}
% {}
% {0pt}
{\thesubsection}
{8pt}
{\scshape}
{}


\begin{document}

\title{Seed Incubation Plant}

\author{
    \IEEEauthorblockN{Anjana Roy}
    \IEEEauthorblockA{\emph{Dept. of Electronics \& Communication}\\ \emph{Rajiv Gandhi Institute of Technology}\\ Kerala, India\\ 22bl14920@rit.ac.in}
    \and
    \IEEEauthorblockN{Aswatheertha T T}
    \IEEEauthorblockA{\emph{Dept. of Electronics \& Communication}\\ \emph{Rajiv Gandhi Institute of Technology}\\ Kerala, India\\ 22bl14661@rit.ac.in}
    \linebreakand
    % \and
    \IEEEauthorblockN{Athira Madhusoodanan}
    \IEEEauthorblockA{\emph{Dept. of Electronics \& Communication}\\ \emph{Rajiv Gandhi Institute of Technology}\\ Kerala, India\\ 22bl14998@rit.ac.in}
    \and
    \IEEEauthorblockN{Daniel V Mathew}
    \IEEEauthorblockA{\emph{Dept. of Electronics \& Communication}\\ \emph{Rajiv Gandhi Institute of Technology}\\ Kerala, India\\ 22bl14682@rit.ac.in}
}

\maketitle

\begin{abstract}
    \noindent The \textbf{Seed Incubation Plant} with \textbf{EMCU} and \textbf{GMU}
    aims to optimize the \emph{germination} and \emph{growth
    process} of \emph{seeds} by providing a \emph{controlled} and
    \emph{monitored environment}.
    %%
    Manual methods are prone to \emph{human error}, requiring constant
    attention to environmental factors which can \emph{fluctuate} and
    \emph{adversely affect seed growth} and may lead to \emph{suboptimal
    results}.
    %%
    In most cases \emph{migrating} to an automated system is not
    \emph{economically feasible}, as it may require importing various
    equipments.Thus, an \emph{indigenous} implementation using a widely
    available \emph{controller} is an alternative solution.
    %%
    This system aims to realize the \emph{environment monitoring and control
    system} which will \emph{monitor} and \emph{control} various aspects that
    influence the optimal growth of the seedling, including \textbf{humidity},
    \textbf{temperature}, \textbf{lighting} and \textbf{soil moisture} by using ESP32
    \emph{microcontroller}. Wifi capabilities of ESP32 ensures
    the accumulation of collected data and presenting it to the user, visually.
    %%
    Along with this, we aims to develop a \emph{growth monitoring unit} powered by a
    fully functional \textbf{TinyML} model that can utilize the limited resources of
    the ESP32 CAM module, enabling close monitoring of different stages of
    seed's germination. In addition, we aim to develop SINC - a fully functional
    companion app to the Seed Incubator.
    %%
    This solution not only tries to improve the \emph{efficiency} and the
    \emph{success rate of seed germination} but also tries to reduce the
    \emph{labour costs} and \emph{energy consumption}, hoping to offer a
    sustainable and reliable alternative to traditional incubation methods.
\end{abstract}

\section{Introduction}

\end{document}
